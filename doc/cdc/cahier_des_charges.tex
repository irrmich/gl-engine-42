\documentclass [a4 paper,11pt]{article}
\usepackage [francais]{babel}
\usepackage [utf8]{inputenc}
\usepackage [T1]{fontenc}
\usepackage{textcomp}
\usepackage{vmargin}
\setmarginsrb{2.5cm}{0.5cm}{2.5cm}{1,5cm}{1,5cm}{0.5cm}{1.0cm}{2.0cm}


\title {Cahier des charges : glEngine}
\author {Gaël Jochaud du Plessix\\
Loick Michard}
\date {2012}

\begin{document}
\maketitle

\section{Rendu 3D}

Le rendu 3D est une des parties importantes d'un bon moteur 3D.
Celui-ci doit permettre la création de scènes réalistes, avec un rendu graphique se rapprochant le plus possible de la réalité.
Pour cela il doit implémenter un ensemble de caractéristiques permettant un rendu rapide et conforme à la réalité.
De plus une des contraintes du moteur est de ne jamais descendre en dessous de 24 FPS.

\subsection{Rasterization}
La rasterization doit être effectuée en utilisant les fonctionnalités de la dernières version d'OpenGL 4.2.

\subsection{Fenêtre d'affichage}
La librairie glEngine doit pouvoir ouvrir une fenêtre permettant d'afficher le rendu 3D du moteur.

\subsection{Effets graphiques}

Elle doit implémenter la plus grande partie des fonctionnalités ci-dessous.

\subsubsection*{Lumière ambiante}
Ajout d'une couleur constante à tous les objets, simulant une luminosité ambiante.
\subsubsection*{Lumière directionnelle}
Gestion de l'éclairage de la scène par une ou plusieurs lumières directionnelles.
\subsubsection*{Lumière ponctuelle}
Gestion de l'éclairage de la scène par une ou plusieurs lumières ponctuelles.
\subsubsection*{Lumière spéculaire}
Reflet de la lumière sur les objets donnant un effet de brillance. Cette couleur peut être différente de la couleur de la lumière pour plus de réalisme.
\subsubsection*{Ombres}
Certains types de lumières doivent produire des ombres sur les objets. 
Pour les objets statiques elle ne doivent pas être recalculées à chaque itération, contrairement aux objets dynamiques afin d'optimiser les performances.
\subsubsection*{Cubemap}
Il doit y voir la possibilité de définir un environnement via une texture. 
Il sera implémenté sous forme de cube map entourant la scène.
Le moteur doit être capable de refléter ou réfracter cette cubemap dans n'importe quel objet.
\subsubsection*{Occlusion ambiante}
On doit pouvoir activer l'occlusion ambiante pour obtenir un effet d'illumination globale.
Elle doit être calculée une seule fois et non à chaque rendu.
\subsubsection*{Bump mapping}
Grâce à une texture spéciale définissant le relief d'un objet, le moteur doit pouvoir donner un effet de déformation et de relief sur l'objet.
\subsubsection*{Réflection}
Dans un premier temps les objets doivent pouvoir refléter l'environnement.
Le moteur pourra plus tard implémenter la réflection de la scène complète dans l'objet.
\subsubsection*{Réfraction}
Dans un premier temps les objets doivent pouvoir réferacter l'environnement.
\subsubsection*{Transparence}
Grâce à une propriété de transparence propre à chaque objet, le moteur doit donner la possibilité de voir la totalité du reste de la scène à travers un objet par transparence.

\section{Gestion de la scène}
Grâce au moteur, on doit pouvoir créer une scène complète comportant tout un ensemble d'éléments.
\subsection{Modèles}

\subsubsection*{Création de mesh}
L'utilisateur doit pouvoir définir un ensemble de point, accompagné d'un ensemble de normale qui formeront un mesh.
\subsubsection*{Importation}
Un modèle doit pouvoir être importé à partir d'un format standard (.obj, .3ds, ...).
Chaque format doit être géré au maximum afin de supporter toutes les fonctionnalités de celui-ci.
On doit ensuite pouvoir récupérer chaque objet composant un modèle séparément.
\subsubsection*{Texture}
Chaque objet doit avoir la possibilité d'être texturé.
Chaque face doit pouvoir définir ses propres coordonnées dans la texture.
\subsubsection*{Hiérarchie}
Tous les objets doivent être organisés dans un arbre hiérarchique.
Ainsi quand on applique une modification sur un parent, elle doit être appliquée à tous ses descendants.

\subsection{Lumières}
On doit pouvoir ajouter différents types de lumières à la scène.

\subsection{Caméras}
La scène doit posséder une caméra courante.
On doit pouvoir ajouter plusieurs caméras et changer la caméra utilisée pour le rendu.

\subsubsection*{Caméra perspective}
Le moteur doit implémenter une caméra avec une projection en perspective, que l'on peut paramétrer avec un angle de vue,
un ratio, une distance minimale et maximale.

\subsubsection*{Caméra orthogonale}
Le moteur doit implémenter une caméra avec une projection orthogonale.

\subsubsection*{Cible caméra}
Une caméra doit pouvoir être définie avec une cible de vue.
Elle regardera alors dans sa direction.

\subsection{Positionnement objets}
Tous les objets doivent être positionnables dans l'espace, sur tous les axes.

\subsubsection*{Position}
On doit pouvoir leur donner une position relative à leur parent.

\subsubsection*{Rotation}
Leur rotation doit pouvoir se faire dans toutes les directions, via un axe de rotation et un angle.

\subsubsection*{Échelle}
Chaque objet peut être mis à l'échelle, indépendemment sur chaque axe.

\subsection{Objets statiques et dynamiques}
Chaque objet doit posséder une propriété statique ou dynamique.
Elle permet d'éviter des recalculs sur les objets statiques et le mouvement des objets dynamiques.

\section{Traitement des métadonnées}
Pour optimiser les performances du moteur il devra posséder un prétraitement des métadonnées.
Il devra être effectué en utilisant les technologies OpenGL et/ou OpenCL.

\subsection{Culling}
\subsubsection*{Elimination des faces cachées}
Le moteur doit être capable d'éliminer de la scène à rendre les faces cachées de tous les objets.
\subsubsection*{Frustrum culling}
Le moteur doit être capable d'éliminer de la scène à rendre les faces en dehors de l'angle de vue.

\subsection{Division spatiale}
Afin d'optimiser la recherche de faces, le moteur devra implémenter une répartition des faces dans l'espace optimisant la recherche.

\subsection{Physique}
Dans une future version le moteur pourra implémenter un ensemble de fonctionnalités permettant une gestion basique de la physique (gravité, collision, ...).

\subsection{Modèles simplifiés}
Une amélioration du moteur pourra permettre de charger des modèles simplifiés si les objets sont éloignés de la caméra.

\section{API}
L'API doit être simple d'utilisation et permettre la plus grande liberté.
Elle doit allier puissance et simplicité en encapsulant les fonctionnalités nécessaires d'OpenGL 4.2 et OpenCL.

\section{Démonstrations}
Le logiciel devra être livré avec un ensemble de démonstrations visant à présenter l'ensemble des fonctionnalités du moteur.

\end{document}
